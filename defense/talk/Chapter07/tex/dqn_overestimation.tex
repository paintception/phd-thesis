\begin{figure}[ht!]
  \begin{tikzpicture}[scale = 0.65]
      \begin{axis}[
	name=ax1,
      	grid style={dashed,gray},
      	grid = both, 
      	tick style=black,
	title=Pong,
        xlabel=Training Steps,
        ylabel=Value Estimates,
      ]


      \addlegendentry{DQN-True Value}
      \addlegendentry{DQN-Estimated Value}
      
      \addplot [ultra thick, black, mark=.] table [y=true_return_DQN, x=training_steps]
      {./Results/Chapter07/logs/overestimation_pong.txt};
      \addplot [ultra thick, blue, mark=.] table [y=estimate_DQN, x=training_steps]{./Results/Chapter07/logs/overestimation_pong.txt};
     
      \legend{}

      \end{axis}

      \begin{axis}[
	at={(ax1.south east)},
	xshift=2cm,
      	grid style={dashed,gray},
      	grid = both, 
      	tick style=black,
	title=Enduro,
        xlabel=Training Steps,
        ylabel= Value Estimates,
	legend columns=3, 
        legend style={font=\Large, at={(-0.85,-0.3,-0.4)},anchor=north west,legend columns=3},
      ]

      \addlegendentry{DQN-True Value}
      \addlegendentry{DQN-Estimated Value}

      \addplot [ultra thick, black, mark=.] table [y=true_return_DQN, x=training_steps]
      {./Results/Chapter07/logs/overestimation_enduro.txt};
      \addplot [ultra thick, blue, mark=.] table [y=estimate_DQN, x=training_steps]{./Results/Chapter07/logs/overestimation_enduro.txt};
 
      \end{axis}
	\end{tikzpicture}
	\caption{Results investigating the extent to which the DQN algorithm suffers from the overestimation bias of the $Q$ function. We can observe that on the \texttt{Pong} environment during the early stages of training, the $\underset{a \in \cal A}{\max}\:Q(s_{t+1}, a)$ estimates quickly grow, while on the \texttt{Enduro} game the values estimated by the $Q$ network keep indefinitely growing, therefore making the algorithm significantly diverge from the real return that is obtained by a trained agent.}
	\label{fig:overestimation_bias_results_dqn} 
\end{figure}

