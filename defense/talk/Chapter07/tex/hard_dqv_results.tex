\begin{figure}[ht!]
  \begin{tikzpicture}[scale = 0.65]
      \begin{axis}[
	name=ax1,
      	grid style={dashed,gray},
      	grid = both, 
      	tick style=black,
	title=Boxing,
        xlabel=Episodes,
        ylabel=Reward,
      ]


      \addlegendentry{DQV}
      \addlegendentry{Hard-DQV}
      
      \addplot [ultra thick, red, mark=.] table [y=DQV, x=episodes]
      {./Results/Chapter07/logs/hard_dqv_boxing_results.txt};
      \addplot [ultra thick, brown, mark=.] table [y=Hard-DQV, x=episodes]{./Results/Chapter07/logs/hard_dqv_boxing_results.txt};
     
      \legend{}

      \end{axis}

      \begin{axis}[
	at={(ax1.south east)},
	xshift=2cm,
      	grid style={dashed,gray},
      	grid = both, 
      	tick style=black,
	title=Pong,
        xlabel=Episodes,
        ylabel= Reward,
	legend columns=3, 
        legend style={font=\Large, at={(-0.65,-0.3,-0.4)},anchor=north west,legend columns=3},
      ]

      \addlegendentry{DQV}
      \addlegendentry{Hard-DQV}
 
      \addplot [ultra thick, red, mark=.] table [y=DQV, x=episodes]
      {./Results/Chapter07/logs/hard_dqv_pong_results.txt};
      \addplot [ultra thick, brown, mark=.] table [y=Hard-DQV, x=episodes]{./Results/Chapter07/logs/hard_dqv_pong_results.txt};
 
      \end{axis}
	\end{tikzpicture}
	\caption{Our results which aim to approximate the $V$ and the $Q$ function with a unique, shared paramterized network, an approach that is heavily inspired by multi-task learning studies that can be found in supervised learning \cite{caruana1997multitask,zhong2020fine,mormont2020multi}. We can see that this extension of DQV, named Hard-DQV, significantly underperforms the original DQV-Learning algorithm.}
	\label{fig:hard_dqv_results} 
\end{figure}

