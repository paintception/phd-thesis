\chapter{Transfer Learning}
\label{ch:transfer_learnimg}

\begin{remark}{Outline}
	We now present Transfer Learning (TL), a machine learning methodology that aims at creating algorithms that are capable of retaining and reusing previously learned knowledge when getting trained on new, unseen problems. Most of the contributions presented within this dissertation are motivated by TL, therefore, we now introduce the reader to this specific learning paradigm with the goal of providing him/her with all the preliminary knowledge that is necessary for fully understanding the research that will be presented in the coming chapters. We start with a gentle introduction to TL in Sec. \ref{sec:tl_introduction} where we present the main concepts underlying TL and explain why it is desirable to have machine learning models that are transferable. We then show in Sec. \ref{sec:rationale} some practical, high level, examples that visually represent the benefits that come from adopting TL strategies in machine learning. We will then provide more rigorous, mathematical definitions of TL in Sec. \ref{sec:definitions} where we will characterize TL both for supervised learning as for reinforcement learning. In Sec. \ref{sec:literature_review} we will thoroughly review how TL has already been studied by the machine learning community, we will again do this both for supervised learning as for reinforcement learning, with an additional focus on how TL is performed when neural networks are used. We end this chapter with Sec. \ref{sec:relevance} where we summarize the most relevant TL concepts that underpin the contributions that are presented in this dissertation.  
\end{remark}

\section{Introduction}
\label{sec:tl_introduction}


\section{Rationale of Transfer Learning}
\label{sec:rationale}

Before mathematically defining TL we start by building some intuition and visually represent how one can observe the benefits of TL in practice. We assume that we would like to solve a certain task, and that we can choose between two different models: a model that has never dealt with any kind of task before, and a second model, which is identical to the first one with the major difference being that it has already seen a similar task in the past. Because of their different nature, we refer to the first model as a `scratch" model, and to the second model as a `pre-trained" model. Ideally, as mentioned in the previous section, we would like the latter model to perform better on the considered task that the first one. Yet, how can we asses if one model is better than the other one? As initially presented by and later we would like the performance of a pre-trained model to result into three possible improvements. If at least one of these improvements is observed while training,  we can then consider the pre-trained model better than the scratch-model. These improvements are the following:
\begin{itemize}
	\item \textcolor{RoyalBlue}{Jumpstart Improvements}: this is a scenario where by the end of training there are no significant differences between the performance of a pre-trained model and the performance of a random model. However this changes when considering the very first training iteration. If jumpstart improvements appear we can usually observe that when both models start their training process, the performance of the pre-trained model is much closer to the one that will be obtained by the end of training than the one of the scratch model. We visually report an example of this scenario in the first plot of Fig. \ref{fig:tl_examples}. In this case the goal is to train a model such that by the end of training its performance will be of $\approx -100$. We can clearly see that by the end of training both models are able to successfully achieve this goal, but that at the very first training iteration, the performance of the pre-trained model is significantly closer to the desired final performance ($\approx -250$) than the one of the scratch model ($\approx -450$). 
	\item \textcolor{RoyalBlue}{Asymptotic Improvements}
	\item \textcolor{RoyalBlue}{Learning Speed Improvements}
\end{itemize}


\begin{figure}[ht!]
  \begin{tikzpicture}[scale = 0.65]
      \begin{axis}[
	name=ax1,
      	grid style={dashed,gray},
      	grid = both, 
      	tick style=black,
	title=Learning Speed Improvements,
        xlabel=Training Iterations,
        ylabel=Performance,
      ]


      \addlegendentry{Pre-trained Model} 
      \addlegendentry{Scratch Model}  
      
      \addplot [ultra thick, blue, mark=.] table [y=Transfer, x=episodes]
      {./Results/Chapter03/logs/speed_improvements.txt};
       \addplot [ultra thick, orange, mark=.] table [y=No-Transfer, x=episodes]
      {./Results/Chapter03/logs/speed_improvements.txt};
     
      \legend{}

      \end{axis}

      \begin{axis}[
      	at={(ax1.south east)},
	xshift=2cm,
	grid style={dashed,gray},
      	grid = both, 
      	tick style=black,
	title=Jumpstart Improvements,
        xlabel=Training Iterations,
        ylabel=Performance,
      ]



      \addlegendentry{Pre-trained Model} 
      \addlegendentry{Scratch Model}  
      
      \addplot [ultra thick, blue, mark=.] table [y=Transfer, x=episodes]
      {./Results/Chapter03/logs/jumpstart_improvements.txt};
       \addplot [ultra thick, orange, mark=.] table [y=No-Transfer, x=episodes]
      {./Results/Chapter03/logs/jumpstart_improvements.txt};
     
      \legend{}

      \end{axis}


      \begin{axis}[
	at={(ax1.south east)},
	xshift=-2cm,
      	yshift=-8cm,
	grid style={dashed,gray},
      	grid = both, 
      	tick style=black,
	title=Asymptotic Improvements,
        xlabel=Training Iterations,
        ylabel= Performance,
	legend columns=4, 
        legend style={font=\Large, at={(-0.3,-0.3,-0.4)},anchor=north west,legend columns=3},
      ]


      \addlegendentry{Pre-trained Model} 
      \addlegendentry{Scratch Model}  
 
      \addplot [ultra thick, blue, mark=.] table [y=Transfer, x=episodes]
      {./Results/Chapter03/logs/asymptotic_improvements.txt};
      \addplot [ultra thick, orange, mark=.] table [y=No-Transfer, x=episodes]
      {./Results/Chapter03/logs/asymptotic_improvements.txt};
       
      \end{axis}
	\end{tikzpicture}
	\caption{A visualization of the three possible desired outcomes that can come from adopting Transfer Learning strategies as initially defined by \citet{langley2006transfer} and later by \citet{lazaric2012transfer}.}
	\label{fig:tl_examples} 
\end{figure}




\section{Mathematical Definitions}
\label{sec:definitions}

\subsection{Supervised Learning}

\begin{definition}
	A domain $\mathcal{D}$ is the combination between an input space $\mathcal{X}$ and a marginal distribution $P(X)$, $\mathcal{D} = \{\mathcal{X},P(X)\}$ where $X$ denotes an instance set defined as $X=\{\vec{x}|\vec{x_i}\in \mathcal{X}, i =1, \cdots, n \}$.
\end{definition}


\begin{definition}
A task $\mathcal{T}$ consists of a label space $\mathcal{Y}$ and a decision function $f$, i.e., $\mathcal{T}=\{\mathcal{Y},f\}$. The decision function $f$ is implicit and can only be learned by sampling data from $\mathcal{X}$.
\end{definition}


\begin{definition}
Given one, or more, observations corresponding to $m^s \in \mathds{N}^{+}$ source domain(s) and tasks(s) (i.e., $\{(\mathcal{D}_{S}_{i}, \mathcal{T}_{S}_{i}|i=1\cdots,m^s)\})$ and some additional observation(s) about $m^T \in \mathds{N}^{+}$ target domain(s) and task(s) (i.e. $\{(\mathcal{D}_{T}_{j},\mathcal{T}_{T}_{j}|j=1,\cdots,m^T)\})$, transfer learning utilizes the knowledge implied in the source domains to improve the performance of the learned decision functions $f^{T}_j(j=1,\cdots,m^T)$ on the target domain(s).
\end{definition}

\paragraph{Inductive Transfer Learning}
\begin{definition}
	Given a source domain $\mathcal{D}_S$ and a learning task $\mathcal{T}_S$, and a target domain $\mathcal{D}_T$ and a learning task $\mathcal{T}_T$, inductive transfer learning aims to help improving the target predictive function $f_T(\cdot)$ in $\mathcal{D}_T$ by using the knowledge in $\mathcal{D}_S$ and $\mathcal{T}_S$, where $\mathcal{T}_S \neq \mathcal{T}_T$. 
\end{definition}



\paragraph{Transductive Transfer Learning}

\begin{definition}
	Given a source domain $\mathcal{D}_S$ and a learning task $\mathcal{T}_S$, and a target domain $\mathcal{D}_T$ and a learning task $\mathcal{T}_T$, transductive transfer learning aims to help improving the target predictive function $f_T(\cdot)$ in $\mathcal{D}_T$ by using the knowledge in $\mathcal{D}_S$ and $\mathcal{T}_S$, where $\mathcal{D}_S \neq \mathcal{D}_T$ and $\mathcal{T}_S = \mathcal{T}_T$. 
\end{definition}



\paragraph{Unsupervised Transfer Learning}

\begin{definition}
	Given a source domain $\mathcal{D}_S$ and a learning task $\mathcal{T}_S$, and a target domain $\mathcal{D}_T$ and a learning task $\mathcal{T}_T$, unsupervised transfer learning aims to help improving the target predictive function $f_T(\cdot)$ in $\mathcal{D}_T$ by using the knowledge in $\mathcal{D}_S$ and $\mathcal{T}_S$, where $\mathcal{T}_S \neq \mathcal{T}_T$ and $\mathcal{Y}_S$ and $\mathcal{Y}_T$ are not observable. 
\end{definition}




\subsection{Reinforcement Learning}

\begin{definition}
Other definition

\end{definition}


\paragraph{Transfer of Value Functions}
\paragraph{Transfer of Features}
\paragraph{Transfer of Policies}

\section{Literature Review}
\label{sec:literature_review}

\subsection{Transfer Learning in Supervised Learning}
\subsection{Transfer Learning in Reinforcement Learning}
\subsection{Deep Transfer Learning}

\section{Relevance for this Dissertation}
\label{sec:relevance}
