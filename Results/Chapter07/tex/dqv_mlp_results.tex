\begin{figure}[ht!]
  \begin{tikzpicture}[scale = 0.65]
      \begin{axis}[
	name=ax1,
      	grid style={dashed,gray},
      	grid = both, 
      	tick style=black,
	title=Acrobot,
        xlabel=Episodes,
        ylabel=Reward,
      ]


      \addlegendentry{DQV}
      \addlegendentry{DQN}
      \addlegendentry{DDQN}
      
      \addplot [ultra thick, red, mark=.] table [y=DQV, x=episodes]
      {./Results/Chapter07/logs/acrobot_results.txt};
      \addplot [ultra thick, blue, mark=.] table [y=DQN, x=episodes]{./Results/Chapter07/logs/acrobot_results.txt};
      \addplot [ultra thick, green, mark=.] table [y=DDQN, x=episodes]{./Results/Chapter07/logs/acrobot_results.txt};
     
      \legend{}

      \end{axis}

      \begin{axis}[
	at={(ax1.south east)},
	xshift=2cm,
      	grid style={dashed,gray},
      	grid = both, 
      	tick style=black,
	title=Cartpole,
        xlabel=Episodes,
        ylabel= Reward,
	legend columns=3, 
        legend style={font=\Large, at={(-0.65,-0.3,-0.4)},anchor=north west,legend columns=3},
      ]

      \addlegendentry{DQV}
      \addlegendentry{DQN}
      \addlegendentry{DDQN}
 
      \addplot [ultra thick, red, mark=.] table [y=DQV, x=episodes]
      {./Results/Chapter07/logs/cartpole_results.txt};
      \addplot [ultra thick, blue, mark=.] table [y=DQN, x=episodes]{./Results/Chapter07/logs/cartpole_results.txt};
      \addplot [ultra thick, green, mark=.] table [y=DDQN, x=episodes]{./Results/Chapter07/logs/cartpole_results.txt};
 
      \end{axis}
	\end{tikzpicture}
	\caption{Our preliminary results that show the benefits in terms of convergence time that can come from jointly approximating the $V$ function alongside the $Q$ function. We can observe that the DQV-Learning algorithm yields faster convergence when compared to popular algorithms which only approximate the $Q$ function: DQN and DDQN.}
	\label{fig:mlp_dqv_results} 
\end{figure}

