\begin{figure}[ht!]
  \begin{tikzpicture}[scale = 0.65]
      \begin{axis}[
	name=ax1,
      	grid style={dashed,gray},
      	grid = both, 
      	tick style=black,
	title=Pong,
        xlabel=Training Steps,
        ylabel=Value Estimates,
      ]


      \addlegendentry{DDQN-True Value}
      \addlegendentry{DDQN-Estimated Value}
      
      \addplot [ultra thick, black, mark=.] table [y=true_return_DDQN, x=training_steps]
      {./Results/Chapter07/logs/overestimation_pong.txt};
      \addplot [ultra thick, green, mark=.] table [y=estimate_DDQN, x=training_steps]{./Results/Chapter07/logs/overestimation_pong.txt};
     
      \legend{}

      \end{axis}

      \begin{axis}[
	at={(ax1.south east)},
	xshift=2cm,
      	grid style={dashed,gray},
      	grid = both, 
      	tick style=black,
	title=Enduro,
        xlabel=Training Steps,
        ylabel= Value Estimates,
	legend columns=3, 
        legend style={font=\Large, at={(-0.85,-0.3,-0.4)},anchor=north west,legend columns=3},
      ]

      \addlegendentry{DDQN-True Value}
      \addlegendentry{DDQN-Estimated Value}
      

      \addplot [ultra thick, black, mark=.] table [y=true_return_DDQN, x=training_steps]
      {./Results/Chapter07/logs/overestimation_enduro.txt};
      \addplot [ultra thick, green, mark=.] table [y=estimate_DDQN, x=training_steps]{./Results/Chapter07/logs/overestimation_enduro.txt};
 
      \end{axis}
	\end{tikzpicture}
	\caption{Results investigating the extent to which the DDQN algorithm suffers from the overestimation bias of the $Q$ function. We can observe that compared to the analysis presented in Fig. \ref{fig:overestimation_bias_results_dqn}, the DDQN algorithm prevents its Q-Network from diverging since on both \texttt{Atari} environments the $\underset{a \in \cal A}{\max}\:Q(s_{t+1}, a)$ estimates do not diverge from the observed real return of a trained agent. These results replicate the findings reported by \citet{van2018deep_triad}.}
	\label{fig:overestimation_bias_results_ddqn} 
\end{figure}

