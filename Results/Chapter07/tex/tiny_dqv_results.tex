\begin{figure}[ht!]
  \begin{tikzpicture}[scale = 0.65]
      \begin{axis}[
	name=ax1,
      	grid style={dashed,gray},
      	grid = both, 
      	tick style=black,
	title=Bank Heist,
        xlabel=Episodes,
        ylabel=Reward,
      ]


      \addlegendentry{DQV}
      \addlegendentry{tiny-dqv}
      
      \addplot [ultra thick, red, mark=.] table [y=DQV, x=episodes]
      {./Results/Chapter07/logs/tiny_dqv_bankheist_results.txt};
      \addplot [ultra thick, black, mark=.] table [y=Tiny-dqv, x=episodes]{./Results/Chapter07/logs/tiny_dqv_bankheist_results.txt};
     
      \legend{}

      \end{axis}

      \begin{axis}[
	at={(ax1.south east)},
	xshift=2cm,
      	grid style={dashed,gray},
      	grid = both, 
      	tick style=black,
	title=Crazy Climber,
        xlabel=Episodes,
        ylabel= Reward,
	legend columns=2,
        legend style={font=\Large, at={(-0.65,-0.3,-0.4)},anchor=north west,legend columns=2},
      ]

      \addlegendentry{DQV}
      \addlegendentry{tiny-dqv}
 
      \addplot [ultra thick, red, mark=.] table [y=DQV, x=episodes]
      {./Results/Chapter07/logs/tiny_dqv_crazyclimber_results.txt};
      \addplot [ultra thick, black, mark=.] table [y=Tiny-dqv, x=episodes]{./Results/Chapter07/logs/tiny_dqv_crazyclimber_results.txt};
 
      \end{axis}
	\end{tikzpicture}
	\caption{Learning curves obtained when reducing the capacity of the convolutional networks that approximate the $V$ and the $Q$ functions. We can observe that albeit each value function is approximated with its own parametrized network, the tiny-dqv extension still yields worse performance. These results highlight that it is not sufficient to simply have two separate neural networks for DQV to perform well, but that a crucial role in DQV's performance is played by the capacity of the networks that are used as well.}
	\label{fig:tiny_dqv_results} 
\end{figure}

