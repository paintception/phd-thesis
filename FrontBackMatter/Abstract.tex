% Abstract

%\renewcommand{\abstractname}{Abstract} % Uncomment to change the name of the abstract

\pdfbookmark[1]{Abstract}{Abstract} % Bookmark name visible in a PDF viewer

\begingroup
\let\clearpage\relax
\let\cleardoublepage\relax
\let\cleardoublepage\relax

\chapter*{Abstract}
Throughout our lifetime we constantly need to deal with unforeseen events, which sometimes can be so overwhelming to look insurmountable. A common strategy that humans as well as animals have learned to adopt throughout millions of years of evolution, is to start tackling novel, unseen situations by re-using knowledge that in the past resulted in successfull solutions. Being able of recognizing patterns across similar settings, as well as the capacity of re-using and potentially adapting an already established skill set, is a crucial component in human's and animal's intelligence. This capacity comes with the name of Transfer Learning. 

The field of Artificial Intelligence (AI) aims to create computer programs that are able of mimicking at least to a certain exent the properties underlying natural intelligence. It follows that among such properties, there is also that of being capable of learning how to solve new tasks whilst exploiting some previously acquired knowledge. Within the mathematical and algorithmic AI toolbox, Convolutional Neural Networks (CNNs) are nowadays by far among the most successfull techniques when it comes to machine learning problems involving high-dimensional and spatially organized inputs. In this dissertation we focus on studying their transfer learning properties, and investigate whether such models can get transferred and trained across a large variety of domains and tasks.

In the quest of better characterizing the transfer learning potential of CNNs, we focus on two of the most common machine learning paradigms: supervised learning and reinforcement learning. After a first part (Part I) devoted to presenting all the necessary machine learning background, we will move to Part II, where the transfer learning properties of CNNs will be studied from a supervised learning perspective. Here we will focus on several computer vision tasks that range from image classification to object detection, which will be tackled by regular CNNs as well as by pruned models. Next, in part III, we will shift our transfer learning analysis to the reinforcement learning scenario. Here we will first start by introducing a novel family of deep reinforcement learning algorithms, and then move towards studying their transfer learning properties alongside that of several other popular model-free reinforcement learning algorithms. 

Our transfer learning experiments allow us to identify the benefits, as well as some of the possible drawbacks that can come from adapting transfer learning strategies, while at the same time shading some light on how convolutional neural networks work. 

\endgroup			

\vfill
